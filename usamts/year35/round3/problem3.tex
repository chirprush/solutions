\begin{solution}{3}
    \textbf{Problem 3.} After a little bit of playing around and perhaps some
    computer simulation (a small exercise in DP actually), one gets a very good
    indicator of the answer, so now it's left to actually make it rigorous.

    First, notice that we can represent the state of the game as a pair \( (t, w) \), where \( t \) represents the number of 2's on the whiteboard and \( w \) represents the number of 1's.

    \begin{definition}
        We call a state \( (t, w) \) \textit{winning} if, no matter what
        sequence of moves one plays, the player who's turn it is with that
        state can win. Similarly, we call a state \textit{losing} if, no matter
        what sequence of moves one plays, the player who's turn it is with the
        state can never win.
    \end{definition}

    In order to determine whether or not Lizzie has a winning strategy, it
    suffices to determine for which \( n \) the state \( (n, 0) \) is winning.
    We shall show a slightly stronger result to do so:

    \begin{claim}
        We have the following:
        \begin{itemize}
            \item If \( n \equiv 0 \pmod{3} \), then \( (n, 0) \) is losing and \( (n, w) \) is winning for \( w \ge 1 \).
            \item If \( n \equiv 1 \pmod{3} \), then \( (n, 0) \) is winning, \( (n, 1) \) is losing, and \( (n, w) \) is winning for \( w \ge 2 \).
            \item If \( n \equiv 2 \pmod{3} \), then \( (n, w) \) is winning for \( w \ge 0 \).
        \end{itemize}
    \end{claim}

    \begin{proof}
        We shall prove this through strict induction.

        Suppose our claim holds true for \( n = 0, 1, 2, \ldots, 3k - 3, 3k - 2, 3k - 1 \). We shall then prove our claim for \( n = 3k, 3k + 1, 3k + 2 \), tackling these cases in order:

        \begin{itemize}
            \item We must first prove that \( (3k, 0) \) is losing. First,
                observe that playing the first move type, which yields the
                state \( (3k - 1, 0) \) is winning for the other player by our
                inductive assumption. Next, notice that \( (3k - w, w) \) for
                \( w \ge 1 \) (which represents all game states reachable by
                the second move type) is always winning for the other player
                because \( (3k - 1, 1) \) is winning by our inductive
                hypothesis and \( (n, w) \) for \( w \ge 2 \) is always
                winning. Thus, there is no way for the current player to reach
                a position where the opponent loses, so \( (3k, 0) \) loses.

                Notice then that, if one considers positions \( (3k, w) \) for
                \( w \ge 1 \), the current player can simply subtract \( w \)
                1's from to force the opponent to \( (3k, 0) \), which is
                losing for them. Therefore, all states of the form \( (3k, w)
                \) for \( w \ge 1 \) are winning.

            \item Next, we shall look at \( (3k + 1, 0) \). This is trivially
                winning because, if we remove a single 2, the position becomes
                \( (3k, 0) \), which we've proved is losing for the other
                player.

                Now we consider the position \( (3k + 1, 1) \). From this
                position, we can play a move of the first type to reach \( (3k,
                1) \), but we have already proven that this is winning for the
                opponent. Using the second move type, we may arrive at
                positions of the form \( (3k + 1 - t, 1 + t) \) and \( (3k + 1
                - t, t) \) for \( t \ge 1 \), but because all positions (with a
                lesser number of twos of course) of the form \( (a, b) \) for
                \( b \ge 2 \) are winning for the opponent, so we need only
                look at \( (3k + 1 - 1, 1) \), which we've already proven to be
                winning for the opponent. Thus, \( (3k + 1, 1) \) is losing.

                It follows that, if the state is of the form \( (3k + 1, w) \),
                where \( w \ge 2 \), the current player may subtract away \( (w
                - 1) \) 1's in order to reach \( (3k + 1, 1) \), so all of these
                positions are winning.

            \item Finally, consider the state \( (3k + 2, 0) \). Since we can
                turn one of the 2's into a 1, the opponent receives \( (3k + 1,
                1) \), which we've proven to be losing, so \( (3k + 2, 0) \) is
                winning.

                For the added 1's cases, we can do something of a very similar
                nature. Consider the states \( (3k + 2, w) \) for \( w \ge 1
                \). By subtracting one from a single 2 and taking away all other \(
                w \) 1's, we can still reach \( (3k + 2 - 1, w - w + 1) = (3k +
                1, 1) \), which means that these positions are winning too.
        \end{itemize}
        Now for our base case, it suffices to prove that \( (0, 0) \) is losing
        and states \( (1, 0), (2, 0) \) are winning, with the states with extra
        1's appended on following exactly according to the logic detailed in
        the above three cases.
        \begin{itemize}
            \item For \( (0, 0) \), notice that if the current player reaches
                this position, then the other player must necessarily have
                reached the state where everything is \( 0 \) first, so this
                means that the state is losing.
            \item The state \( (1, 0) \) is trivially winning, for one may
                simply take away the 2 to win.
            \item The state \( (2, 0) \) is also winning. Once again using
                logic previously detailed in the top three cases, we can see
                that \( (1, 1) \) is losing, so we can subtract one from a
                single 2 to send the opponent to a losing position.
        \end{itemize}
    \end{proof}

    Thus, it follows through induction that, after proving this slightly stronger result, the only \( n \) for which Lizzie can always win are \( n \) of the form \( 3k + 1 \) and \( 3k + 2 \).

    % First we must show that if n is divisible by 3 then we always lose, which
    % is a bit hard and might need induction? I don't actually know how I'm
    % going to tackle this.

    % We're probably going to have to go with some variant of strict induction
    % where we start with \( n = 0, n = 1, n = 2 \) and then prove, from the
    % three previous solutions, the three forward solutions.

    % n = 0: loses trivially
    % n = 1: wins trivially
    % n = 2: wins pretty trivially

    % n = 3: loses because
    % a) Going from 222 -> 22 loses trivially because n - 1 wins for the other
    %    person
    % b) Going from 222 -> 221 loses because 221 wins for other (we'll have to prove that I suppose
    % c) Going from 222 -> 211 loses because 211 wins for other (we'll also have to prove that)
    % d) Going from 222 -> 111 loses because 111 trivially wins

    % Game states: 
    % 0 1 1 1 1 1 1 1 1 1 
    % 1 0 1 1 1 1 1 1 1 1 
    % 1 1 1 1 1 1 1 1 1 1 
    % 0 1 1 1 1 1 1 1 1 1 
    % 1 0 1 1 1 1 1 1 1 1 
    % 1 1 1 1 1 1 1 1 1 1 
    % 0 1 1 1 1 1 1 1 1 1 
    % 1 0 1 1 1 1 1 1 1 1 
    % 1 1 1 1 1 1 1 1 1 1 
    % 0 1 1 1 1 1 1 1 1 1

    % If n = 0 mod 3, one wins if there is at least 1 '1'
    % - This is the easiest to prove if we have that n = 0 mod 3 always loses for 0 '1's
    % If n = 1 mod 3, one wins if there are 0 '1's or at least 2 '1's
    
    % If n = 2 mod 3, any number of '1's added on will win
\end{solution}
