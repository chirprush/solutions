\begin{solution}{2}
    \textbf{Problem 2.} We shall search for possible unordered pairs \( (a, b, c) \) and display all possible products at the end of the answer.

    Notice that, of all \( abc \) cubes, \( (a-2)(b-2)(c-2) \) are unpainted.
    Thus we are looking for all unordered \( (a, b, c) \) such that
    \[
        f(a, b, c) := \frac{(a-2)(b-2)(c-2)}{abc} = \frac{1}{5}
    ,\]
    where \( a, b, c \ge 3 \).

    This is a standard looking Diophantine equation, to which we seemingly want
    to apply SFFT and then look at prime factorizations for solutions. Treating
    \( c \) as a constant and removing the denominators, we achieve
    \begin{align*}
        5(a-2)(b-2)(c-2) &= abc \\
        (2c-5)ab - 5(c-2)a - 5(c-2)b + 10(c-2) &= 0 \\
        (2c-5)^2 ab - 5(c-2)(2c-5) a - 5(c - 2)(2c - 5)b + 10(c-2)(2b-5) &= 0 \\
        \left( (2c - 5)a - 5(c-2) \right)\left( (2c - 5)b - 5(c-2) \right) &= 5c(c-2)
    .\end{align*}

    With this, we must go through values of \( c \) to find solutions in \( a
    \) and \( b \), which makes the proof slightly inelegant and a little
    bashy, but oh well. First, however, we must prove a very handy claim, which allows us to stop looking after \( c = 20 \).
    \begin{claim}
        WLOG suppose \( c = \max\{a, b, c\} \). There exists no \( a, b, c \) for which \( f(a, b, c) = 1/5 \) when \( c > 20 \).
    \end{claim}
    \begin{proof}
        Suppose \( c > 20 \). Observe that
        \[
            f(a, b, 20) < f(a, b, c) < \lim_{c' \to \infty} f(a, b, c')
        ,\]
        so if \( f(a, b, c) = 1/5 \), then we must have that
        \[
            \frac{9}{10} \cdot \frac{(a-2)(b-2)}{ab} < \frac{1}{5} < \frac{(a-2)(b-2)}{ab}
        .\]
        If both \( a \ge 4 \) and \( b \ge 4 \), then this condition cannot possibly be true. Thus, WLOG we set \( b = 3 \) to look for any remaining cases that may have a solution. Doing so, we get that
        \[
            \frac{a-2}{a} > \frac{3}{5} \implies a > 5
        \]
        and simulatenously
        \[
            \frac{a-2}{a} < \frac{2}{3} \implies a < 6,
        \]
        which cannot be true.

        Because we've covered all cases, there exists no solution for \( c \ge 20 \) (or any of the dimensions being greater than \( 20 \)).
    \end{proof}
    The only real bashy part of the proof comes into play with checking all
    cases where \( c = 3, 4,\ldots, 20 \). Luckily, we can use a bit of Python
    to do the work for us:

    \verbatiminput{scripts/grogg-answers/main.py}
    This outputs
    \verbatiminput{scripts/grogg-answers/output.txt}

    Thus, the only cases that yield any new solutions are \( c = 3 \) and \( c = 4 \). (that is,
    there solutions for \( c = 5, 6, \ldots, 20 \), but they do not yield a
    different product from any of the cases found in cases \( c = 3 \) and \( c
    = 4 \)).

    So, the only possible numbers of cubes are the products of the values in each unordered pair: \( \boxed{120, 160, 240, 360} \).
\end{solution}
