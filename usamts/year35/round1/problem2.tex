\begin{solution}{2}
    \textbf{Solution 2.} No; this number can never be prime.

    Suppose the sequence \( p_0, p_1, \ldots, p_{100} \) is some permutation of
    the list of numbers 101 numbers \( 2024, 2025, \ldots, 2124 \). We can
    express all concatenations of the numbers as the following sum, with the
    order determined by the sequence \( p_k \):
    \[
        P = \sum_{k = 0}^{100} p_k 10^{4k} = \sum_{k = 0}^{100} p_k 100^{2k}
    .\]
    Take \( P \) modulo \( 101 \).
    \begin{align*}
        P &\equiv \sum_{k = 0}^{100} p_k (-1)^{2k} \pmod{101} \\
        &\equiv \sum_{k = 0}^{100} p_k \pmod{101} \\
        &\equiv \sum_{k = 2024}^{2124} k \pmod{101}
    .\end{align*}
    That is, regardless of the ordering of the numbers, \( P \) is equivalent
    to the sum of list of 101 numbers when modulo 101. Using the formula for the sum of the
    first \( n \) natural numbers, we can clearly see that
    \begin{align*}
        P &\equiv \frac{2124 \cdot 2125}{2} - \frac{2023 \cdot 2024}{2} \pmod{101} \\
        &\equiv 1062 \cdot 2125 - 2023 \cdot 1012 \pmod{101} \\
        &\equiv 52 \cdot 4 - 3 \cdot 2 \pmod{101} \\
        &\equiv 6 - 3 \cdot 2 \pmod {101} \\
        &\equiv 0 \pmod{101}
    .\end{align*}
    Any way of concatenating the numbers will end up with the same result,
    meaning that all numbers of the questioned form are divisible by \( 101 \)
    and thus not prime.
\end{solution}
