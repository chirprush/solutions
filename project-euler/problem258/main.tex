\documentclass[a4paper, 12pt]{article}

\usepackage{chirpstyle}

\begin{document}

\section*{Project Euler: Problem 258}

\begin{sidebox}
    \begin{problem}
        Define a sequence \( g_k \) by
        \[
            g_{k + 2000} = g_k + g_{k + 1}
        ,\]
        and \( g_k = 1 \) for \( 0 \le k \le 1999 \).

        \vspace{0.3cm}

        Find \( g_K \bmod 20092010 \), where \( K := 10^{18} \).
    \end{problem}
\end{sidebox}

\begin{solution}
    We may model the linear recurrence with the following \( 2000 \times 2000 \) matrix:
    \[
        T = \begin{bmatrix}
            0 & 1 & 0 & 0 & \cdots & 0 \\
            0 & 0 & 1 & 0 & \cdots & 0 \\
            \vdots & \vdots & \vdots & \ddots & \vdots & \vdots \\
            0 & 0 & 0 & 0 & \cdots & 1 \\
            1 & 1 & 0 & 0 & \cdots & 0
        \end{bmatrix}
    .\]
    Effectively, if we have some input vector
    \[
        \vec{v} = \begin{bmatrix}
            g_k \\
            g_{k + 1} \\
            \vdots \\
            g_{k + 1999}
        \end{bmatrix}
    ,\]
    applying our matrix \( T \) transitions to the next state. In other words,
    \[
        T \vec{v} = \begin{bmatrix}
            g_{k + 1} \\
            g_{k + 2} \\
            \vdots \\
            g_{k + 2000}
        \end{bmatrix}
    .\]
    Thus, we can rephrase our problem as finding the value of the state vector given by
    \[
        T^{K} \begin{bmatrix}
            1 \\
            1 \\
            \vdots \\
            1
        \end{bmatrix} \pmod{20092010}
    ,\]
    where the modulus implies element-wise modular arithmetic.

    An initial reaction to this formulation may be either fast matrix exponentiation or diagonalization, however the first fails because even with the time complexity of \( O(n^3 \log_2{K}) \), it is still slow, and the second fails due to floating point errors. What we can do, however, is simplify things just a bit.

    Interestingly enough, we can show that taking powers of this matrix forms a group, and in particular, a cyclic group isomorphic to the integers modulo \( 20092010 \). Indeed, since \( \det T = -1 \), this holds.

    Equipped with this, we can use some useful tools from number theory. In particular, one may observe that
    \[
        T^{K} \equiv T^{K \bmod{\lambda(20092010)}} \pmod{20092010}
    ,\]
    where \( \lambda(n) \) represents the Carmichael function. Using this, we drastically decrease the exponent, as \( \lambda(20092010) = 2006004 \). In particular, this tells us that
    \[
        g_{K} \equiv g_{K \bmod{2006004}} \equiv g_{939940} \pmod{20092010}
    .\]
    This works out well, as now we can simply calculate the value from the linear recurrence in suitable time (specifically \( O(\lambda{(M)}) \), where \( M \) is the modulus).

    \textcolor{WildStrawberry}{This seems to be incorrect and the underlying reason why perchance has to do with like the using number theory for matrices part.} Empirically it seems that for an \( n \times n \) matrix that has a non-zero determinant in \( \mathbb{Z}_p \), where \( p \) is prime is
    \[
        \mathrm{ord} (T) = p^{2^{n - 2}}
    ,\]
    which makes no sense :sob: but also since this grows way too fast the entire approach isn't going to be useful at all noooooo.
\end{solution}

\end{document}
