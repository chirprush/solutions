\documentclass[a5paper, 10pt]{article}

\usepackage{chirpstyle}

\begin{document}

\section*{Project Euler: Problem 323}

\begin{chirpbox}
\begin{problem}
    Let \( y_1, y_2, \ldots \) be a sequence of random unsigned \( 32 \)-bit integers that are uniformly chosen. Define a sequence \( (x_k) \) such that \( x_1 = y_1 \) and \( x_k = x_{k - 1} \mid y_k \) (bitwise OR).

    \vspace{0.3cm}

    It can be seen that there will eventually be a smallest index \( N \) such that \( x_N = 2^{32} - 1 \). Find the expected value of \( N \) to \( 10 \) digits after the decimal point.
\end{problem}
\end{chirpbox}

\begin{solution}
    When I first saw this problem, I thought it was harder than it looked, but once you start thinking in terms of random variables, it actually becomes quite easy.
    \begin{definition}
        Define a \textit{bit random variable} \( y \) to be a discrete random variable which takes on the values \( 0 \) and \( 1 \) with equal probability \( 1/2 \).
    \end{definition}
    We can decompose each \( y_k \) into bit random variables, which we shall notate as follows:
    \[
        y_k = (y_k[0], y_k[1], \ldots, y_k[31])
    ,\]
    where \( y_k[i] \) represents the \( i \)th bit of \( y_k \). This is motivated by the fact that, since each \( y_k \) is uniformly distributed, we can treat it as a combination of independent, similarly uniform, random variables. We can decompose each \( x_k \) in the same fashion.

    There are two important observations to make:
    \vspace{-0.3cm}
    \begin{itemize}[topsep=0pt]
        \item The probability that \( x_k = 2^{32} - 1 = 11\ldots1_2 \) can be found by taking the intersections of the events of the individual bits being \( 1 \). In other words,
            \[
                P(x_k = 2^{32} - 1) = \prod_{i} P(x_k[i] = 1)
            .\]
            By symmetry, we can simplify this further by noticing that the probability that any one of these bits are \( 1 \) is the exact same. So,
            \[
                \prod_{i} P(x_k[i] = 1) = P(x_k[0] = 1)^{32}
            .\]

        \item The probability \( P(x_k[i] = 1) \) is rather easy to calculate. Observe that
            \[
                P(x_k[i] = 1) = 1 - P(x_k[i] = 0) = 1 - \frac{1}{2^k}
            .\]
    \end{itemize}
    With this, the expected value of \( N \), the smallest index, is easy to find. The probability that some index \( n \) is the smallest such index is the difference between the probability that \( x_n \) is \( 2^{32} - 1 \) and the probability that \( x_{n-1} \) is \( 2^{32} - 1 \). Thus
    \[
        E[N] = \sum_{n = 1}^{\infty} n \left( (1 - 2^{-n})^{32} - (1 - 2^{-(n-1)})^{32} \right)
    .\]
    One can very easily truncate this to about \( 50 \) terms to get the answer to \( 10 \) decimal places.
\end{solution}

\end{document}
