\documentclass[a4paper, 12pt]{article}

\usepackage{chirpstyle}

\begin{document}

\section*{USAMO 2004 Solutions}

\setcounter{pnum}{1}

\begin{chirpbox}
    \begin{problemnum}
        Suppose \( a_1, a_2, \ldots, a_n  \) are integers whose greatest common divisor is \( 1 \). Let \( S \) be a set of integers with the following properties:
        \begin{enumerate}[label=\textbf{\alph*.}]
            \item For each \( a_i \), \( a_i \in S \).
            \item For each pair \( (a_i, a_j) \) (where the indices are not neessarily distinct), \( a_i - a_j \in S \).
            \item For each pair \( (x, y) \) where \( x, y \in S \), \( x + y \in S \implies x - y \in S \).
        \end{enumerate}
    \end{problemnum}
\end{chirpbox}

\begin{solution}
    We may first easily prove that \( 0, 1 \in S \).
    \begin{itemize}
        \item For the first case of \( 0 \in S \), this trivially holds because we can take \( a_1 - a_1 = 0 \), which is in \( S \) by the second rule.
        \item For the second case of \( 1 \in S \), we may see that, sincethe greatest common divisor of all elements is \( 1 \), there must exist an odd element in \( a_1, a_2, \ldots, a_n \), which is also in \( S \) by the first condition. We shall write this odd number in the form \( 2k + 1 \) and use the third condition to see that since \( (k + 1) + k = 2k + 1 \in S \), then \( (k + 1) - k = 1 \in S \). \textcolor{red}{Wait no this doesn't work because you can't know that \( k, k + 1 \in S \) whoops.}
    \end{itemize}

    We shall also show that \( x \in S \implies -x \in S \). This follows trivially from the fact that \( x = 0 + x \implies 0 - x = -x \in S \).
    
    From here, finishing is trivial, as we may induct up. More specifically, we have that \( x \in S \implies x + 2 \in S \), and since we have both \( 0 \) and \( 1 \) in \( S \), this reaches all integers after reflection (because \( x \in S \implies -x \in S \)). We may prove this proposition by observing that
    \[
        x = (x + 1) + (-1) \in S \implies (x + 1) - (-1) = x + 2 \in S
    ,\]
    which finishes the proof when plugging in \( x = 0, 1, \ldots \) and so on.
\end{solution}

\end{document}
