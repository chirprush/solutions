\documentclass[a4paper, 12pt]{article}

\usepackage{chirpstyle}

\begin{document}

\section*{Codeforces Problem 1946F}

\begin{chirpbox}
    \begin{problem}
        Given a permutation of length \( n \) and for \( q \) queries
        consisting of integers \( l, r \), determine the number of sorted
        sequences of indices \( t_1, t_2, \ldots, t_k \) for \( k \ge 1 \) such
        that each \( t_i \) lies between \( l \) and \( r \), and \( a_{t_i} \)
        divides \( a_{t_{i+1}} \).
    \end{problem}
\end{chirpbox}

\begin{solution}[\textcolor{red}{Incorrect}]
    We shall start by precomputing some helpful variables (the time
    complexities for these shall be verified later).
    \begin{itemize}
        \item Let \( \textsf{starts}[i] \) denote the number of sequences that
            start with \( i \).
        \item Let \( \textsf{ends}[i] \) denote the number of sequences that
            end with \( i \).
        \item Let \( \textsf{starts\_prefix}[i] \) be the prefix of \(
            \textsf{starts} \) and \( \textsf{ends\_prefix} \) the prefix of \(
            \textsf{ends} \). In other words,
        \begin{align*}
            \textsf{starts\_prefix}[i] &= \sum_{j \le i} \textsf{starts}[j], \\
            \textsf{ends\_prefix}[i] &= \sum_{j \le i} \textsf{ends}[j]
        .\end{align*}
    \end{itemize}
    Once we have these arrays precomputed, we may easily determine the number
    of sequences that satisfy our conditions. For a given interval of indices
    \( l, r \), the answer is
    \[
        (\textsf{starts\_prefix}[r] - \textsf{starts\_prefix}[l-1]) - (\textsf{ends\_prefix}[n] - \textsf{ends\_prefix}[r])1
    .\]
    It is now sufficient to determine a method to compute \( \textsf{starts} \)
    and \( \textsf{ends} \) in sufficient time.

    We shall start by determining \( \textsf{starts} \). We find this rather
    simply by iterating backwards (note that the order of this is important)
    for each \( a_i \) and checking the positions of all multiples of \( a_i \)
    to see if they lie in strictly greater indices and then adding path counts
    accordingly. More formally,
    \[
        \textsf{starts}[i] = 1 + \sum_{\substack{a_i \mid a_j, \\ j > i}} \textsf{starts}[j]
    .\]

    One can verify that this runs in sufficient time as
    \[
        \sum_{i = 1}^{n} \left\lfloor \frac{n}{i} \right\rfloor = O(n \log{n})
    .\]

    The computation of the \( \textsf{ends} \) array is slightly different, and this asymmetry owes itself to the computational differences in finding divisors as opposed to the simpler multiples. Thus, we must take a different approach. We have that
    \[
        \textsf{ends}[i] = 1 + \sum_{\substack{a_j \mid a_i, \\ j < i}} \textsf{ends}[j]
    .\]
    Iterating over each value of \( i \) linearly would result in us having to factor each \( a_i \), but luckily we can simply over \( j \), resulting in another \( O(n \log{n}) \) pass over the array.

    It actually appears that this solution is incomplete and incorrect, as
storing the starting path and ending path prefix counts does not allow one to
uniquely determine all paths (the equation given above for the answer in terms
of the prefixes is obviously wrong after more thought is given). Another
solution path will have to be attempted.

    This being said, it is likely that the true solution follows a similar path, especially with the \( O(n \log{n}) \) computation times.
\end{solution}

\begin{solution}
\end{solution}

\end{document}
