\documentclass[a4paper, 12pt]{article}

\usepackage{chirpstyle}

\begin{document}

\section*{Codeforces Problem 1916H1}

\begin{sidebox}
    \begin{problem}
        For each \( r \) where \( 0 \le r \le m \), count the number of \( n \times n \) matrices over \( \mathbb{F}_p \) with rank \( r \).
    \end{problem}
\end{sidebox}

\begin{solution}
    After quite a bit of thinking and structuring the problem, we arrive at the following realization, which allows us to somewhat easily solve for the quantities we're looking for: for each \( 0 \le r \le m \),
    \[
        \sum_{k = 0}^{r} \textsf{sub}[r, k] \cdot \textsf{ranks}[k] = p^{nr}
    ,\]
    where \( \textsf{sub}[r, k] \) denotes the number of dimension \( k \) subspaces that a dimension \( r \) vector space over \( \mathbb{F}_p \) has, and \( \textsf{ranks}[k] \) denotes the number of \( n \times n \) matrices with rank \( k \) formed using vectors from a fixed \( k \) dimensional vector space. Our answer for each \( r \) is then \( \textsf{sub}[n, r] \cdot \textsf{ranks}[r] \).

    In other words, we have a system of \( n + 1 \) equations in our \( n + 1 \) unknowns that we wish to solve for, namely each \( \textsf{ranks}[k] \). Thus, we've reduced the problem down to counting the number of subspaces of a vector space over this finite field.

    \begin{claim}
        We have that
        \[
            \textsf{sub}[r, k] = \frac{(p^r - 1)(p^r - p) \cdots (p^r - p^{k - 1})}{(p^k - 1)(p^k - p) \cdots (p^k - p^{k-1})}
        .\]
    \end{claim}

    \begin{proof}
        We would like to be able to pick out \( k \)-dimensional subspaces of a \( r \)-dimensional vector space, so we do so by picking \( k \) linearly independent vectors to form a basis. However, this overcounts by quite a bit, so we must divide by the number of ways to choose \( k \) linearly independent vectors from the subspace that we found.
    \end{proof}

    Since the denominator of \( \textsf{sub}[r, k] \) can be precomputed in reasonable time and the numerator can be computed with an easy recursion, we can solve for \( \textsf{ranks}[k] \) and thus find the answer in something like \( O(k^2 \log n) \) time, where the logarithm factor appears because of calculating the exponentials modulo the fixed prime \( P \).
\end{solution}

\end{document}
