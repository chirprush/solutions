\documentclass[a4paper, 12pt]{article}

\usepackage{chirpstyle}

\begin{document}

\section*{Codeforces Problem 1948E}

\begin{chirpbox}
    \begin{problem}
        Suppose we have a graph with \( n \) vertices and a permutation \( a \) of the numbers \( 1 \) through \( n \), and we add an edge on the graph for every pair of vertices \( (i, j) \) if
        \[
            | i - j | + | a_i - a_j | \le k
        .\]
        Determine the permutation \( a \) for which the number of distinct cliques in the graph is a minimum.
    \end{problem}
\end{chirpbox}

\begin{remark}
    A \textit{clique} is defined to be a complete subgraph of a graph.
\end{remark}

\begin{constraints}
    We have that
    \begin{itemize}
        \item \( 1 \le t \le 1600 \),
        \item \( 2 \le n \le 40 \),
        \item \( 1 \le k \le 2n \)
    \end{itemize}
\end{constraints}

\begin{solution}
    We shall first tackle the problem of creating a clique of size \( m \). Immediately, we see that \( m \le \min\{n, k\} \).

    \begin{claim}
        A clique of size \( m \) is constructible with cost \( m \) at the very least.
    \end{claim}
    \begin{proof}
        Observe first that any cost for a clique of size \( m \) cannot be lower than \( m \) because the maximum value of \( |i - j| \) is achieved when \( i = 1 \) and \( j = m \), and \( |a_i - a_j| \ge 1 \), so their sum must be greater than or equal to \( m \) for some pair of vertices \( (i, j) \).

        We provide the following construction for a permutation that costs only \( m \) and results in a clique of size \( m \):
        \[
            a: \left\lceil m / 2 \right\rceil, (\left\lceil m / 2 \right\rceil - 1), \ldots, 1, m, (m-1), \ldots, (m + 1 - \left\lceil m / 2 \right\rceil)
        ,\]
        though because the math is a tad gruesome, we shall not verify here that the sequence does in fact fulfill the cost conditions stated above.
    \end{proof}

    Now since we have that a clique of size \( m \) is constructible and \( m \le \min\{n, k\} \), the solution is rather obvious (simply form as many cliques of greatest size that we can). We also trivially have that when \( k \ge n \), the minimum clique number is \( 1 \), which seems to be a rather good sanity check. In particular, we also have a closed form for the minimum number of cliques needed:
    \[
        \textsf{cliques} = \left\lceil \frac{n}{\min\{n, k\}} \right\rceil
    .\]
\end{solution}
\end{document}
