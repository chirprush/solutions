\documentclass[a4paper, 12pt]{article}

\usepackage{chirpstyle}

\begin{document}

\section*{Codeforces Problem 1967A}

\begin{chirpbox}
    \begin{problem}
        Let \( n \) be some fixed natural number. We are given frequency counts of integers \( 1 \) to \( n \) as well as the choice of \( k \) more counts.

        \vspace{0.3cm}

        Let the \textit{score} of some arrangement of numbers denote the number of subarrays of the arrangement that are a permutation of the numbers \( 1 \) to \( n \) and let \( \textsf{counts} \) be some arrangement of the given and chosen numbers as mentioned above. Find the maximum score of \( \textsf{counts} \) for which it is optimally chosen.
    \end{problem}
\end{chirpbox}

\begin{solution}
    While I shall not try to prove that it is the most optimal as that would take a lot of thinking, we have decent intuition to believe that a sliding window technique achieves the highest score. Let \( c[i] \) denote the frequency counts given to us and \( s[i] \) denote the additional \( k \) chosen counts. We then have the following claim.

    \begin{claim}
        Let the optimal maximal minimal shared value obtained by choice of \( s[i] \) be
        \[
            \textsf{value} = \max_{s[i]} \min_{i} \Bigl( c[i] + s[i] \Bigr)
        .\]
        We then have that the maximal score is
        \[
            1 + (n-1)(\textsf{value}) + \varepsilon
        ,\]
        where \( \varepsilon \) is a residual term that may range from \( 0 \) to \( 2n - 1 \) inclusive.
    \end{claim}
\end{solution}

\end{document}
