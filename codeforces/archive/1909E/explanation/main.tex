\documentclass[a4paper, 12pt]{article}

\usepackage{chirpstyle}

\begin{document}

\section*{Codeforces Problem 1909E}

\begin{chirpbox}
    \begin{problem}
        Suppose we have \( n \) lamps labelled \( 1 \) to \( n \) which are all initially turned off. We may then pick a set of indices \( S \) of size at least \( 1 \), with a set of \( m \) constraints for which if \( u_i \in S \) then \( v_i \in S \). For each index \( i \) in the chosen \( S \), all lamps which have indices that are multiples of \( i \) are toggled.

        \vspace{0.2cm}

        Determine any choice (if there exists one) of \( S \) such that, after all lamps have been toggled according to the indices, there are at most \( \lfloor n / 5 \rfloor \) lamps that are on.
    \end{problem}
\end{chirpbox}

\begin{solution}
    We shall first get some simple cases out of the way that fall out from direct observation or a bit of playing around with the problem.
    \begin{claim}
        For \( n < 5 \), it is always impossible.
    \end{claim}
    \begin{proof}
        This follows trivially from the fact that, under these conditions \( \lfloor n / 5 \rfloor = 0 \). This is impossible to achieve when we are picking indices from the set. (Actually I'm not sure the best way to prove this though).
    \end{proof}

    \begin{claim}
        For \( n \ge 20 \), it is always possible to achieve such a configuration, and we may simply choose all indices.
    \end{claim}

    \begin{proof}
        For this, we shall show that the number of lamps that are on after choosing all indices is strictly less than or equal to \( \lfloor n / 5 \rfloor \) for all \( n \ge 5 \).

        Let \( f(n) \) denote the number of lamps on after choosing the set \( S = \{ 1, 2, \ldots, n \} \). Observe that
        \begin{align*}
            f(n) &= f(n-1) + \sum_{d \mid n} 1 \bmod 2 \\
            &= f(n-1) + d(n) \bmod 2
        ,\end{align*}
        where \( d(n) \) denotes the number of divisors of \( n \). If one recalls that for some number \( k = p_1^{e_1} p_2^{e_2} \cdots p_n^{e_n} \),
        \[
            d(k) = (e_1 + 1)(e_2 + 1) \cdots (e_n + 1)
        ,\]
        we can see that \( d(n) \) is only odd when all exponents are even. This means that \( d(n) \mod 2 \) is the same as the indicator function for whether \( n \) is a perfect square. Since \( f(1) = 1 \), we can solve this recurrence to see that
        \[
            f(n) = \lfloor \sqrt{n} \rfloor
        ,\]
        which means that this problem is trolling, but also since \( \lfloor \sqrt{n} \rfloor < \lfloor n / 5 \rfloor \) for \( n \ge 20 \), we can essentially ignore this case when problem solving.
    \end{proof}

    This tells us that the only bounds for which we have to actually solve this problem are when \( 5 \le n < 20 \). This means that we can actually use an \( O(2^n) \) solution or even perhaps \( O(n 2^n) \), which is funny if not a little stupid.

    We shall try for an \( O(n 2^n) \) solution when \( n < 20 \). For each index \( i \), precompute a bitmask, denoted \( \textsf{bitmask}(i) \), that corresponds to the lights toggled after including \( i \) in the set. One may see that for some set of indices \( S = \{ i_1, i_2, \ldots, i_k \} \), the resulting lamps chosen are given by the following bitmask:
    \[
        \textsf{bitmask}(S) := \bigoplus_{j = 1}^{k} \textsf{bitmask}(i_k)
    ,\]
    where \( \oplus \) denotes bitwise XOR. Thus, the number of lamps toggled are the number of ones in the resulting value of \( \textsf{bitmask}(S) \). Using the fact that having a running bitwise XOR and counting the number of ones bits in a number are theoretically both \( O(1) \) operations, this solution part runs in \( O(2^n) \) time, as we simply just iterate through all subsets Since the problem also gives us the additional condition for elements to include, though, we must handle this and there are at worst \( n - 1 \) pairs for each element, so the final running time is \( O(n 2^n) \).

    Actually, the above solution isn't quite what I have coded (there are some complications in the chaining of conditions which required directed graphs, dfs, and dp, but I'll write about this later).
\end{solution}

\end{document}
