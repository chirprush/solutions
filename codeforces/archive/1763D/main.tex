\documentclass[a4paper, 12pt]{article}

\usepackage{chirpstyle}
\usepackage{svg}
\usepackage{float}

\begin{document}

\section*{Codeforces Problem 1763D}

\begin{sidebox}
    \begin{problem}
        We call a permutation \textit{bitonic} if all the elements increase until a certain index \( k \), where \( 2 \le k \le n - 1 \), and then decrease until the end.

        \vspace{0.3cm}

        Count modulo \( 10^9 + 7 \) the number of bitonic permutations of length \( n \) such that \( B_i = x \) and \( B_j = y \), where \( i < j \) and \( x \ne y \).
    \end{problem}
\end{sidebox}

\begin{solution}
    Given the definition of a bitonic permutation, we are motivated to go casewise on \( k \) when counting. Notice that since the elements increase up to \( k \) and decrease afterwards, \( B_k \) must be equal to the maximum value, \( n \).

    \begin{observation}
        Without the for \( B_i \) and \( B_j \), the number of total bitonic permutations \( A \) on \( n \) numbers is given by
        \[
            A = \sum_{k = 2}^{n - 1} \binom{n - 1}{k - 1} = 2^{n - 1} - 2
        .\]
        This works because we must set \( B_k = n \), giving us \( n - 1 \) elements left to choose from. We can then choose any \( k - 1 \) of the remaining elements because there is only one way to increasingly sort each choice, and the choice of \( k - 1 \) elements uniquely determines the decreasing elements.
    \end{observation}
    With this idea in mind, we can start looking at the constrained problem. There are two main cases that we have to consider:
    \begin{itemize}
        \item Either \( B_i = x = n \) or \( B_j = y = n \). Having this condition forces \( k \) to be either \( i \) or \( j \). This case is actually easier, however, as we only have to consider one value. Consulting the following diagram,
            \begin{figure}[H]
                \centering
                \includesvg[width=350pt]{figures/edgecase1}
            \end{figure}

            we see that the answers for each case respectively should be
            \[
                \binom{n - y - 1}{j - i - 1} \binom{y - 1}{n - j}, \text{ and } \binom{n - x - 1}{j - i - 1} \binom{x - 1}{i - 1} 
            ,\]
            since we must handle the sorting condition and since choosing two of the sections uniquely determines the choice of the third.
        \item Otherwise we must go casewise on all possible values of \( k \). In particular, though, we can group these cases of \( k \) into three further cases: \( k < i \), \( i < k < j \), and \( k > j \).
    \end{itemize}
    We shall now tackle these cases:
    \begin{itemize}
        \item Case \( k < i \). We must have that \( x > y \), and we have the following diagram:
            \begin{figure}[H]
                \centering
                \includesvg[width=350pt]{figures/maincase1}
            \end{figure}

            For this case, it's clear to see that the answer is
            \[
                \binom{n - x - 1}{i - k - 1} \binom{x - y - 1}{j - i - 1}  \binom{y - 1}{n - j}  
            .\]
        \item Case \( i < k < j \). We have the following diagram:
            \begin{figure}[h]
                \centering
                \includesvg[width=350pt]{figures/maincase2}
            \end{figure}

            This case is slightly more complicated, as there is an overlap of elements in the inner two sections. With a bit of thinking, we see that the condition \( n - \min\{x, y\} - 1 \ge j - i - 2 \) must hold (otherwise there is no possible permutation). For convenience, we shall say that WLOG \( x < y \), simplifying the condition to \( n - x - 1 \ge j - i - 2 \).

            We shall first place the elements in range \( y + 1 \) to \( n - 1 \). We must choose \( j - k - 1 \) of them to place in the third interval, leaving the others to be uniquely placed in the second interval. We shall then choose the remaining \( k - i - 1 - (n - y - 1 - (j - k - 1)) = j - i - 1 + y - n \) numbers out of the \( y - x - 1 \) we have. All that is left is to choose the first interval out of simplicity, which then uniquely determines the fourth.

            Thus, if \( x < y \), the answer is
            \[
                \binom{n - y - 1}{j - k - 1}  \binom{y - x - 1}{j - i - 1 + y - n}  \binom{x - 1}{i - 1} 
            ,\]
            and if \( x > y \), the answer is
            \[
                \binom{n - x - 1}{k - i - 1} \binom{x - y - 1}{j - i - 1 + x - n} \binom{y - 1}{n - j} 
            .\]
        \item Case \( k > j \). We must have that \( x < y \), and we have the following diagram:
            \begin{figure}[H]
                \centering
                \includesvg[width=350pt]{figures/maincase3}
            \end{figure}

            It is also apparent for this case that the answer is
            \[
                \binom{n - y - 1}{k - j - 1} \binom{y - x - 1}{j - i - 1} \binom{x - 1}{i - 1} 
            .\]
    \end{itemize}
\end{solution}

\end{document}
