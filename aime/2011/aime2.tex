\documentclass[a5paper, 10pt]{article}

\usepackage{chirpstyle}

\begin{document}

\section*{AIME II 2011 Solutions}

\begin{chirpbox}
\begin{problem}[\href{https://amctrivial.com/?page=2011_AIME_II_Problems/Problem_9}{2019 AIME II \#9}]
    Let \( x_1, x_2, \ldots, x_6 \) be non-negative real numbers such that
    \[
        x_1 + x_2 + x_3 + x_4 + x_5 + x_6 = 1
    ,\]
    and
    \[
        x_1 x_3 x_5 + x_2 x_4 x_6 \ge \frac{1}{540}
    .\]
    Let \( p \) and \( q \) be positive relatively prime positive integers such that \( p/q \) is the maximum possible value of
    \[
        \sum_{cyc} x_1 x_2 x_3
    .\]
    Find \( p + q \).
\end{problem}
\end{chirpbox}

\begin{solution}
    By AM-GM we have that
    \begin{align*}
        \frac{1}{27}(x_1 + x_3 + x_5)^3 &\ge x_1 x_3 x_5, \\
        \frac{1}{27}(x_2 + x_4 + x_6)^3 &\ge x_2 x_4 x_6
    .\end{align*}
    Motivated by this, let \( A := x_1 + x_3 + x_5 \) and \( B := x_2 + x_4 + x_6 \) so that we have
    \begin{align*}
        A + B &= 1, \\
        A^3 + B^3 &\ge \frac{1}{20}
    .\end{align*}
    Cubing our first relation, we get that
    \[
        1 = A^3 + 3A^2 B + 3A B^2 + B^3 = A^3 + B^3 + 3AB \ge 3AB + \frac{1}{20}
    ,\]
    so \( AB \le 19/60 \).

    Note that
    \begin{align*}
        AB &= (x_1 + x_3 + x_5)(x_2 + x_4 + x_6) \\
        &= x_1 x_2 + x_1 x_4 + x_1 x_6 + x_2 x_3 + x_3 x_4 + x_3 x_6 + x_2 x_5 + x_4 x_5 + x_5 x_6
    .\end{align*}

    Consider the cyclic sum given in the problem, which we shall denote \( S \). Observe that
    \begin{align*}
        S &= (x_1 + x_3 + x_5)(x_2 x_3 + x_4 x_5 + x_1 x_6) \\
        &+ (x_2 + x_4 + x_6) (x_3 x_4 + x_5 x_6 + x_1 x_2)
    .\end{align*}
\end{solution}

\end{document}
