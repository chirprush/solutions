\documentclass[a4paper, 12pt]{article}

\usepackage{chirpstyle}

\begin{document}

% Well I thought it would be an interesting problem before I actually solved it
% but it's actually rather easy bruh
\section*{Fun Kattis Math Problem}

\begin{chirpbox}
\begin{problem}
    Define the following function \( g(n) \) to be
    \[
        g(n) = \{ g(k) \colon 0 \le k < n \}
    .\]
    where \( g(0) = \{\} \).

    \vspace{0.3cm}

    Define \( f(n) \) to be the number of braces and commas required to write out \( g(n) \). Derive a closed form expression for \( f(n) \) (i.e. a form that is computable in \( O(1) \) time).
\end{problem}
\end{chirpbox}

\begin{solution}
    We shall start with the obvious: we shall decompose \( f(n) \) into the sum of the number braces and the number of the commas, and then we shall derive a recurrence for both of them. In particular, write
    \[
        f(n) = 2 b(n) + c(n)
    ,\]
    where \( b(n) \) is the number of open brackets and \( c(n) \) the number of commas.

    We shall start with deriving a closed form for \( b(n) \). Observe that
    \[
        b(n) = 1 + \sum_{k = 0}^{n - 1} b(k), \quad b(0) = 1
    .\]
    This then tells us that \( b(n) = 2^n \).

    For \( c(n) \), we have that
    \[
        c(n) = n - 1 + \sum_{k = 0}^{n - 1} c(k), \quad c(0) = c(1) = 0
    .\]
    This then tells us that \( c(n) = 2^{n - 1} - 1 \) for \( n \ge 1 \).

    So our final answer is simply
    \[
        f(0) = 2, \quad f(n) = 2^{n + 1} + 2^{n-1} - 1
    \]
\end{solution}

\end{document}
