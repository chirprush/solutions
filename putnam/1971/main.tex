\documentclass[a4paper, 12pt]{article}

\usepackage{chirpstyle}

\begin{document}

\section*{1971 Putnam Solutions}

\begin{sidebox}
    \begin{problem}[A2]
        Find all polynomials \( f \) such that \( f(0) = 0 \) and \( f(x^2 + 1) = f(x)^2 + 1 \).
    \end{problem}
\end{sidebox}

\begin{solution}
    It's obvious that the polynomial \( f(x) := x \) works, and there don't seem to be any other obvious solutions, so we'll try and prove that this is the only such polynomial.

    In experimenting with plugging in numbers, we see that \( f(x)^2 = f(-x)^2 \) and also the following chain of substitutions holds:
    \begin{align*}
        f(0^2 + 1) = f(1) &= f(0)^2 + 1 = 1 \\
        f(1^2 + 1) = f(2) &= f(1)^2 + 1 = 2 \\
        f(2^2 + 1) = f(5) &= f(2)^2 + 1 = 5 \\
        &\vdots
    \end{align*}
    so it looks like we can construct infinitely many points such that \( f(x) = x \). If \( f \) were any function, this wouldn't help us prove anything, but (unless I'm being smooth-brained) since \( f \) is a polynomial, this actually pretty much just gives us the solution; we just have to formalize this notion. In particular,

    \begin{lemma}
        Fix some natural number \( M \). There exists at least \( M \) distinct values of \( x \) such that \( f(x) = x \).
    \end{lemma}

    \begin{proof}
        We shall prove this with induction, with the base case of \( M = 1 \) being trivial, as \( f(0) = 0 \) is given.

        Suppose we have at least \( M - 1 \) distinct values of \( x \) such that \( f(x) = x \) and that, in particular, \( t \) is the largest such value. We see that
        \[
            f(t^2 + 1) = f(t)^2 + 1 = t^2 + 1
        ,\]
        so clearly \( x = t^2 + 1 \) is a solution to \( f(x) = x \). Since \( t^2 + 1 > t \), this is a new unique solution. Thus, there are at least \( M \) distinct solutions.
    \end{proof}

    This now lets us finish. Suppose \( f \) is of degree at most \( k \) (\( f \) must be of finite degree, otherwise it wouldn't be a polynomial). Our lemma above allows us to say that there are \( k + 1 \) points such that \( f(x) = x \). Since \( k + 1 \) points define a polynomial and \( f(x) := x \) satisfies these conditions, this solution is unique, and \( f(x) \) is equivalent to \( x \).
\end{solution}

\end{document}
